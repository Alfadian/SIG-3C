\section{Dini Permata Putri (1174053)}
\subsection{Buku}
Belum Lunas 
\subsection{Pengertian Sistem Informasi Geografis}
\begin{enumerate}
\item definisi sistem informasi geografis 
Sistem Informasi Geografis atau disingkat SIG (bahasa Inggris Sistem Informasi Geografis (SIG) adalah sebuah komputer yang berbasis sistem informasi yang digunakan untuk menyediakan informasi bentuk digital dan menganalisis terhadap permukaan geografi bumi. Sistem Informasi Geografis (SIG) diartikan sebagai sistem untuk menyimpan, memantau, mengintegrasi, memanipulasi, menganalisis dan memaparkan data yang berkaitan dengan semua ruang yang terkait dengan keadaan bumi. Artikel yang berasal dari Prahasta yang membahas tentang GIS adalah menyimpan, membaca, mengintegrasi, memanipulasi, menganalisis dan memaparkan data yang berkaitan dengan semua ruang yang berkaitan dengan keadaan bumi., Informasi dan Sistem 
[1] dan dalam artikel dari Husein dkk, yang menyebutkan bahwa Sistem Informasi Geografis merupakan pemahaman dari Geografi Informasi dan Sistem [2].
karena Sistem Informasi Geografis adalah bidang kajian ilmu dan teknologi yang masih baru. Beberapa resolusi dari Sistem Informasi Geografis yaitu:
Definisi SIG menurut (Rhind, 1988) yaitu GIS adalah sistem komputer untuk mengumpulkan, memeriksa, mengintegrasikan dan menganalisis informasi yang berkaitan dengan permukaan bumi. 
Definisi SIG menurut (Marble dan Peuquet, 1983) dan (Parker, 1988: Ozemoy et al., 1981; Burrough, 1986) yaitu GIS berkaitan dengan data ruang-waktu dan sering tetapi tidak selalu, mempekerjakan perangkat keras dan perangkat lunak komputer.
SIG adalah suatu sistem yang dapat mengupayakan perangkat keras (perangkat keras), perangkat perangkat lunak (perangkat lunak), dan data, serta dapat digunakan dan digunakan sistem penyimpanan, pengolahan, serta analisis data yang dilakukan secara simultan, sehingga dapat diperoleh seluruh informasi yang dimuat langsung dengan aspek ke dalam ruangan.  SIG adalah manajemen data spasial dan data non-spasial yang berbasis komputer dengan menggunakan tiga karakteristik dasar, yaitu: 
\end{enumerate}
\begin{enumerate}
\item Memiliki fenomena yang aktual (data variabel non-lokasi) dan terkait dengan topik topik di lokasi penelitian 
\item merupakan suatu lokasi Tertentu 
\item Memiliki dimensi waktu.  Alasan GIS diperlukan karena data spasial ditanganinya sangat sulit karena peta dan data cepatnya kadaluarsa sehingga tidak ada layanan penyediaan data dan informasi yang diberikan menjadi tidak akurat
\end{enumerate}
Berikut merupakan keistimewaan analisa dengan sistem informasi geografis:
\begin{enumerate}
\item analisa proximity
\item analisa overlay
\end{enumerate}
\subsection{Sejarah}
Peta merupakan penggambaran grafis atau bentuk skala (mempertimbangkan) pada konsep tentang bumi dalam hal ini peta merupakan alat untuk melengkapi atau memuat tentang ilmu kebumian.  Bagaimana peta dahulu ditemukan?  Pengetahuan tentang dasar pembentukan sama seperti filsafat, yang mana sering dianggap berbeda.  Peta Menurut Claudius Ptolemaeus Ptolemy, Claudius Ptolemaeus yang dikenal dengan nama Ptolemy, hidup antara tahun 100 M dan 168 M, beliau merupakan salah satu sarjana sains pada masanya.  Dia tinggal dan bekerja di Alexandria, kota Mesir yang merupakan pusat Intelektual dunia barat dengan perpustakaan paling luas yang pernah diciptakan.  Ptolemy membawa semua pengetahuan dan keterampilan matematika dan astronomi dan menerapkannya pada pembuatan peta.
\subsection{koordinat}
Koordinat digunakan untuk menentukan titik di Bumi melalui garis lintang dan garis bujur.  Koordinat dibagi menjadi dua bagian irisan yaitu irisan melintang yang disebut dengan garis lintang mulai dari khatulistiwa, membesar ke arah kutub (utara maupun selatan) sedangkan yang lain membujur mulai dari garis Greenwhich membesar ke arah barat dan timur.

\subsection{Data geospasial}
data raster adalah data yang tersimpan dalam bentuk grid atau petak jadi terbentuk pada sebuah ruang yang teratur dalam bentuk pixel (elemen gambar).  Foto digital seperti areal fotografi atau satelit merupakan bagian dari data raster pada peta.  Data raster memiliki kisi-kisi data terus. Diharapkan menggunakan gambar berwarna seperti fotografi, yang disetujui dengan tingkat merah, hijau, dan biru pada sel.  Data Raster (atau disebut juga dengan sel grid) merupakan data yang dihasilkan dari sistem penginderaan jauh.  Pada data raster.  Obyek geografis direpresentasikan sebagai struktur sel grid yang disebut dengan pixel (elemen gambar).
\subsection{link}
https://youtu.be/lK9n98oaRHM
