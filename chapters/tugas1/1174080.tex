\section{Handi Hermawan (1174067)}
\subsection{Definisi}
\begin{enumerate}
	\item Geographic information system (GIS) adalah sebuah komputer yang berbasis sistem iformasi digunakan untuk memberikan informasi dalam bentuk digital dan analisa terhadap geografi bumi
	\item Sistem informasi geografis diartikan sebagai sistem untuk menyimpan, memeriksa, mengintegrasi, memanipulasi, mengananlisis dan memaparkan data yang berkaitan dengan keadaan bumi.
	\item GIS adalah manajemen data spasial dan non-spasial yang berbasis komputer dengan menggunakan  tiga karakteristik dasar, yaitu fenomena yang aktual, merupakan kejadian disuatu lokasi tertentu, memiliki dimensi waktu.
	
	\item Ada keistimewaan menganalisa menggunakan sistem informasi geografis yaitu :
\end{enumerate}
	\begin{itemize}
	\item Analisa Proximity adalah geografi yang berbasis pada jarak antar layer
	\item Analisa Overlay adalah proses integrasi data dari lapisan layer yang berbeda (overlay) yang secara analisa membutuhkan lebih dari satu layer.
    \end{itemize}
    
\subsection{Pemahaman GIS}
Geografi objeknya mengacu pada spesifikasi dalam suatu tempat atau ruang. Dimana simbol, warna dan gaya garis digunakan sebagai perwakilan dari setiap spasial yang berbeda pada peta dua dimensi berupa :
    \begin{itemize}
	\item Format titik
	\item Format Garis 
	\item Format Poligon 
	\item Format Permukaan
    \end{itemize}
Informasi yaitu berasal dari kata pengolahan sejumlah data 
Sistem yaitu kumpulan elemen elemen yang saling berintegrasi 
    
\subsection{Komponen GIS}
Komponen GIS terdiri dari lima komponen :
     \begin{itemize}
	\item Sistem komputer (perkakas dalam sistem oprasi) merupaka hardwarenya.
	\item Software GIS merupakan ArcGIS untuk tujuan perancangan, pengurusan, ataupun pemodelan pada kebutuhan tertentu 
	\item Database GIS 
	\item FMethods GIS (prosedur analisis) melibatkan proses input, menyimpan, mengurus, menukar, menganalisis, dan output
	\item People (orang yang menggunakan GIS/User)
	\end{itemize}
\subsection{Model Sistem Informasi Geografis}
GIS mempresentasikan real world dengan data spasial yang terbagi dua model:
    \begin{itemize}
    \item Vektor meresepsikan sebagai mozaik yang terdiri atas garis, polygon, titik dan noders. Berbasiskan pada titik dengan koordinar (x.y) untuk membangun objeck spasialnya.
    \item Raster adalah data yang dihasilkan dari sistem pengindraan yang jauh. Meresepsikan sebagai struktur sel grid yang disebut pixel.
    \end{itemize}

\subsection{Link}
https://youtu.be/wjwKH9jGwV8

